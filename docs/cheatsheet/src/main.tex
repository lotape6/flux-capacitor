\documentclass[10pt,a4paper]{article}
\usepackage[utf8]{inputenc}

\usepackage[landscape,margin=1cm]{geometry}
\usepackage[english]{babel}
\usepackage[T1]{fontenc}
\usepackage[utf8]{inputenc}
\usepackage{tikz}
\usetikzlibrary{shadows,shapes.geometric,calc}

\usepackage{tcolorbox}
\usepackage{graphicx}
\usepackage{hyperref}
\usepackage{graphicx,wrapfig}

\title{ }
\date{}
\usepackage[default]{raleway}
\usepackage{fontawesome}
\usepackage[T1]{fontenc}

\usepackage{hyperref}
\usepackage{enumitem}
\usepackage{lipsum}

\usepackage{xcolor}
\definecolor{customcolor}{HTML}{3d3c38}
\definecolor{alert}{HTML}{CD5C5C}
\definecolor{w3schools}{HTML}{4CAF50}
\definecolor{subbox}{gray}{0.60}
\definecolor{codecolor}{HTML}{FFC300}
\definecolor{greenbash}{HTML}{8AE234}
\definecolor{bg}{HTML}{76096f}
\definecolor{termbg}{HTML}{2d0922}
\definecolor{blues}{HTML}{719ece}
\definecolor{yellow}{HTML}{aabb44}
\definecolor{fluxblue}{HTML}{1E90FF}
\definecolor{fluxgreen}{HTML}{32CD32}
\colorlet{xx}{customcolor}

%--------------------------Editor mode.

% Bibliography functionality removed for simplicity

%----------------------------------------
%--------------------------------------------------------------------------------
\usepackage{tcolorbox}

\tcbuselibrary{most}

\tcbset{tcbox width=auto,left=1mm,top=1mm,bottom=1mm,
right=1mm,boxsep=1mm,middle=1pt}

\newenvironment{mycolorbox}[2]{%
\begin{tcolorbox}[grow to left by=-1em,grow to right by=-1em,capture=minipage,fonttitle=\large\bfseries, enhanced jigsaw,boxsep=1mm,colback=#1!30!white,on line,tcbox width=auto, toptitle=0mm,colframe=#1,opacityback=0.8,nobeforeafter,title=#2]%
}{\end{tcolorbox}\\[0.2em]}

\newenvironment{subbox}[2]{%
\begin{tcolorbox}[capture=minipage,fonttitle=\normalsize\bfseries, enhanced jigsaw,boxsep=1mm,colback=#1!30!white,on line,tcbox width=auto,left=0.3em,top=1mm, toptitle=0mm,colframe=#1,opacityback=0.7,nobeforeafter,title=#2]\footnotesize %
}{\normalsize\end{tcolorbox}\vspace{0.1em}}

\newenvironment{multibox}[1]{%
\begin{tcbraster}[raster columns=#1,raster equal height,nobeforeafter,raster column skip=1em,raster left skip=1em,raster right skip=1em]}{\end{tcbraster}}

\newenvironment{textbox}[1]{\begin{mycolorbox}{customcolor}{#1}}{\end{mycolorbox}}

%-------------------------------
\newenvironment{simplecodebox}[1]{%
\begin{tcolorbox}[colback=codecolor!5,colframe=codecolor!80!black,left=5mm,enhanced,title=#1, fonttitle=\bfseries]%
\ttfamily\footnotesize%
}{%
\end{tcolorbox}%
}

%--------------------------------------------------------------------------------
\newcommand{\punkti}{~\lbrack\dots\rbrack~}

\renewenvironment{quote}
               {\list{\faQuoteLeft\phantom{ }}{\rightmargin\leftmargin}%
                \item\relax\scriptsize\ignorespaces}
               {\unskip\unskip\phantom{xx}\faQuoteRight\endlist}
               

%--------------------------------------------------------------------------------
\newcommand{\bgupper}[3]{\colorbox{#1}{\color{#2}\huge\bfseries\MakeUppercase{#3}}}
\newcommand{\bg}[3]{\colorbox{#1}{\bfseries\color{#2}#3}}

\newcommand{\mycommand}[2]{{\textbf{#1}}~\dotfill{}~{\scriptsize #2}}
\newcommand{\sep}{{\scriptsize~\faCircle{ }~}}

\newcommand{\bggreen}[1]{\medskip\bgupper{w3schools}{black}{#1}\\[0.5em]}
\newcommand{\green}[1]{\smallskip\bg{w3schools}{white}{#1}\\}
\newcommand{\red}[1]{\smallskip\bg{alert}{white}{#1}\\}
\newcommand{\yellow}[1]{\smallskip\bg{yellow}{white}{#1}\\}
\newcommand{\blue}[1]{\smallskip\bg{fluxblue}{white}{#1}\\}

\newcommand*\keystroke[1]{%
  \tikz[baseline=(key.base)]
    \node[%
      draw,
      fill=white!8 ,
      opacity=0.9,
      drop shadow={shadow xshift=0.25ex,shadow yshift=-0.25ex,fill=black,opacity=0.70},
      rectangle,
      rounded corners=2pt,
      inner sep=1pt,
      line width=0.5pt,
      font=\scriptsize\sffamily
    ](key) {\hspace{0.4mm} \textbf{#1}\hspace{0.6mm}\strut}
  ;
}

\usepackage{multicol}
\setlength{\columnsep}{30pt}

\setlength{\parindent}{0pt}
\pagestyle{empty}

\usepackage{csquotes}

\newcommand{\loremipsum}{Lorem ipsum dolor sit amet.}

\makeatletter
\newcommand{\globalcolor}[1]{%
  \color{#1}\global\let\default@color\current@color
}
\makeatother

\AtBeginDocument{\globalcolor{white}}

% Add a subtle gradient background using TikZ
\usepackage{eso-pic}
\AddToShipoutPicture{%
  \begin{tikzpicture}[remember picture,overlay]
    \fill[inner color=fluxblue!3,outer color=fluxgreen!2] (current page.south west) rectangle (current page.north east);
    % Add some decorative dots
    \fill[fluxblue,opacity=0.1] (2,2) circle (0.5cm);
    \fill[fluxgreen,opacity=0.1] (15,8) circle (0.3cm);
    \fill[fluxblue,opacity=0.1] (25,3) circle (0.4cm);
    \fill[fluxgreen,opacity=0.1] (8,15) circle (0.2cm);
    \fill[fluxblue,opacity=0.1] (20,12) circle (0.3cm);
  \end{tikzpicture}%
}

%--------------------------------------------------------------------------------

\begin{document}

\small
\begin{multicols}{3}

\thispagestyle{empty}
\scriptsize

\section{Core Commands}

\begin{textbox}{\href{https://github.com/lotape6/flux-capacitor}{flux connect}}

Create enhanced tmux sessions with automatic directory switching, environment loading, and custom command execution.

\underline{\textbf{Basic Usage}}
\begin{itemize}
    \item \textbf{flux connect <directory>} - Create/attach to session
    \item \textbf{flux connect .} - Create session for current directory
    \item \textbf{flux connect \textasciitilde/projects/myapp} - Absolute path
\end{itemize}

\underline{\textbf{Options}}
\begin{itemize}
    \item \textbf{-n, --session-name <name>} - Custom session name
    \item \textbf{-p, --pre-cmd <command>} - Run command in new panes
    \item \textbf{-P, --post-cmd <command>} - Run after session ends
    \item \textbf{-e, --env-file <path>} - Source environment file
    \item \textbf{-f, --force-new} - Always create new session
\end{itemize}

\green{Pros}

Automatic session reuse based on directory path\\

Environment variables loaded in all new panes\\

Consistent directory switching across all panes\\

Smart session management with unique naming\\

\yellow{Tips}

Use \textbf{--pre-cmd} for activating virtual environments or loading modules\\

Combine \textbf{--env-file} with project-specific .env files\\

Use \textbf{--force-new} for multiple sessions in same directory\\

\end{textbox}

\begin{textbox}{flux launch}

Validate YAML files with comprehensive error reporting.

\underline{\textbf{Usage}}
\begin{itemize}
    \item \textbf{flux launch config.yml} - Validate YAML file
    \item \textbf{flux launch deployment.yaml} - Check deployment file
\end{itemize}

\green{Pros}

Python-based validation with PyYAML\\

Fallback to extension checking if Python unavailable\\

Clear error messages for debugging\\

\yellow{Tips}

Great for CI/CD pipeline validation\\

Use with shell completion for quick file selection\\

\end{textbox}

\begin{textbox}{flux clean}

Reset tmux server and clean up all sessions.

\underline{\textbf{Usage}}
\begin{itemize}
    \item \textbf{flux clean} - Kill all tmux sessions and server
\end{itemize}

\red{Warning}

This will terminate ALL active tmux sessions!\\

Save your work before running this command\\

\yellow{Tips}

Use when tmux becomes unresponsive\\

Good for starting fresh development sessions\\

\end{textbox}

\vfill\null
\columnbreak

\centering
\vspace{6pt}
by:\Large \color{greenbash}lotape6 \scriptsize

\raggedright

\centering

\huge \color{fluxblue}Flux {\color{blues}Capacitor} \color{white} Cheatsheet \scriptsize 

\raggedright
\vspace{5mm}

\section{Advanced Usage}

\begin{textbox}{Session Management}

\underline{\textbf{Session Reuse}}

Flux intelligently reuses existing sessions:
\begin{itemize}
    \item Same directory = attach to existing session
    \item Different session name = create new session
    \item \textbf{--force-new} flag = always create unique session
\end{itemize}

\underline{\textbf{Environment Management}}

\mycommand{-e .env}{Load project environment variables}

\mycommand{-e config/dev.env}{Load development configuration}

\mycommand{export VAR=value}{Variables available in all panes}

\underline{\textbf{Command Execution}}

\mycommand{-p "source venv/bin/activate"}{Python virtual environment}

\mycommand{-p "eval \$(ssh-agent)"}{SSH agent setup}

\mycommand{-P "deactivate"}{Cleanup after session}

\end{textbox}

\begin{textbox}{Shell Completion}

Flux provides smart completion for multiple shells.

\underline{\textbf{Available Completions}}
\begin{itemize}
    \item \textbf{Bash} - flux-completion.bash
    \item \textbf{Zsh} - flux-completion.zsh  
    \item \textbf{Fish} - flux-completion.fish
\end{itemize}

\underline{\textbf{Features}}
\begin{itemize}
    \item Command completion (\keystroke{Tab})
    \item Directory suggestions for connect
    \item YAML file suggestions for launch
    \item Option flag completion
    \item \textbf{fzf} integration when available
\end{itemize}

\green{Pros}

Interactive file selection with preview\\

Smart filtering based on file types\\

Works with bat for syntax highlighting\\

\yellow{Tips}

Install \textbf{fzf} and \textbf{bat} for best experience\\

Use \keystroke{Tab} \keystroke{Tab} to trigger file selector\\

\end{textbox}

\begin{textbox}{Configuration}

\underline{\textbf{Configuration: \textasciitilde/.config/flux/flux.conf}}

\mycommand{FLUX\_VERBOSE\_MODE=true}{Enable detailed logging}

\mycommand{FLUX\_ROOT=\$\{HOME\}/.flux}{Root directory}

\mycommand{FLUX\_LOGS\_DIR=\$\{FLUX\_ROOT\}/logs}{Log file location}

\underline{\textbf{Tmux Integration}}

Custom tmux configuration at \textbf{config/.tmux.conf}

Session hooks for automatic directory switching

Environment variable propagation to new panes

\blue{Configuration}

Modify \textasciitilde/.config/flux/flux.conf for global settings\\

Each session stores its own environment variables\\

\end{textbox}

\vfill\null
\columnbreak

\section{Examples \& Workflows}

\begin{textbox}{Common Workflows}

\underline{\textbf{Development Setup}}

\begin{simplecodebox}{Python Development}
\# Navigate to project and activate environment\\
flux connect \textasciitilde/projects/myapp $\backslash$\\
\phantom{xx}-p "source venv/bin/activate" $\backslash$\\
\phantom{xx}-e .env $\backslash$\\
\phantom{xx}-n myapp-dev\\
\\
\# New panes will automatically:\\
\# - cd to \textasciitilde/projects/myapp\\
\# - source venv/bin/activate\\
\# - load .env variables\\
\end{simplecodebox}

\underline{\textbf{DevOps Workflow}}

\begin{simplecodebox}{Kubernetes Management}
\# Setup for k8s cluster management\\
flux connect \textasciitilde/k8s-configs $\backslash$\\
\phantom{xx}-p "kubectl config use-context prod" $\backslash$\\
\phantom{xx}-e k8s.env $\backslash$\\
\phantom{xx}-n k8s-prod\\
\\
\# Validate deployment files\\
flux launch deployment.yaml\\
flux launch service.yaml\\
\end{simplecodebox}

\underline{\textbf{Multiple Environments}}

\begin{simplecodebox}{Parallel Development}
\# Development environment\\
flux connect \textasciitilde/project -n dev $\backslash$\\
\phantom{xx}-e .env.dev -p "npm run dev"\\
\\
\# Production debugging (force new)\\
flux connect \textasciitilde/project -n prod $\backslash$\\
\phantom{xx}-e .env.prod -f $\backslash$\\
\phantom{xx}-p "npm run build"\\
\end{simplecodebox}

\end{textbox}

\begin{textbox}{Keyboard Shortcuts}

\underline{\textbf{Tmux Navigation (within sessions)}}

\keystroke{Alt} + \keystroke{h} $\rightarrow$ horizontal split

\keystroke{Alt} + \keystroke{v} $\rightarrow$ vertical split

\keystroke{Alt} + \keystroke{Arrow Keys} $\rightarrow$ switch panes

\keystroke{Alt} + \keystroke{Shift} + \keystroke{Arrow} $\rightarrow$ resize panes

\keystroke{Alt} + \keystroke{1-9} $\rightarrow$ select windows 1-9

\keystroke{Ctrl} + \keystroke{b} + \keystroke{r} $\rightarrow$ reload config

\keystroke{Ctrl} + \keystroke{b} + \keystroke{d} $\rightarrow$ detach session

\underline{\textbf{Shell Shortcuts}}

\keystroke{Ctrl} + \keystroke{r} $\rightarrow$ search command history (with fzf)

\keystroke{Ctrl} + \keystroke{t} $\rightarrow$ file finder (with fzf)

\keystroke{Tab} $\rightarrow$ flux command completion

\keystroke{Tab} \keystroke{Tab} $\rightarrow$ interactive file selection

\end{textbox}

\begin{textbox}{Installation \& Setup}

\underline{\textbf{Quick Install}}

\begin{simplecodebox}{Installation}
git clone \\
\phantom{xx}https://github.com/lotape6/flux-capacitor.git\\
cd flux-capacitor\\
./install.sh\\
\end{simplecodebox}

\underline{\textbf{Prerequisites}}
\begin{itemize}
    \item \textbf{tmux} - Session management
    \item \textbf{fzf} - Fuzzy finding (optional)
    \item \textbf{bat} - File preview (optional)  
    \item \textbf{python3} - YAML validation
\end{itemize}

\green{Installation Complete}

The install script automatically configures:\\

Shell completion for bash/zsh/fish\\

Tmux configuration and integration\\

Environment setup and PATH configuration\\

\yellow{Tips}

Installation directory: \textbf{\$\{HOME\}/.flux}\\

All configuration is handled automatically\\

\red{Uninstall}

\textbf{\$FLUX\_ROOT/uninstall.sh} (default: \textasciitilde/.flux/uninstall.sh)

\end{textbox}

\end{multicols}

\end{document}